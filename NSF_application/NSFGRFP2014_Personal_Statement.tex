\documentclass[12pt]{article}

\usepackage[margin=1in]{geometry}

\frenchspacing

\let\OLDthebibliography\thebibliography
\renewcommand\thebibliography[1]{
  \OLDthebibliography{#1}
  \setlength{\parskip}{0pt}
  \setlength{\itemsep}{0pt plus 0.3ex}
}

\begin{document}
\centerline{\bf Personal Statement and Previous Research --- Matthew Le} 


In my first year as an undergrad at the University of Minnesota, I recall hearing about a research group that was developing data driven approaches for furthering our understanding of climate change. I found this work fascinating and for the first time saw what kind of wide-reaching impacts computer science can have on our society. This group had been publishing work on detecting deforestation, land cover change, predicting hurricane patterns, and many other efforts to further our understanding of global warming --- a problem that many consider to be the defining challenge that our generation is facing. Fast forward five years, I am now a PhD student and my goal is that one day an undergrad will think of my work in the same way and encourage them to pursue research as a career as well.

During my first year of graduate school, I began working with my advisor and our collaborators at the University of Chicago on the Manticore project, which extends Standard ML, a popular functional language with various parallel features.  My first project with the group involved working on an analysis performed by the compiler that determines when it is safe to perform higher-order function inlining, a problem that previously could not be efficiently solved due to the complexity of ensuring that the environment at the call site matches the environment captured at the function definition.  Our solution rephrased the problem as a graph reachability issue and allowed us to conservatively estimate when inlining higher-order functions is safe with high accuracy.  We published this work at the ACM SIGPLAN International Conference on Functional Programming (ICFP'14) earlier this year.  

A second project that I worked on in my first year involved extending Manticore with a form of shared state known as IVars.  It had previously been shown that extending a purely functional parallel language with IVars preserves a deterministic semantics.  However, we showed that if you further extend the language with speculative parallelism and cancellation, determinism no longer holds.  In this work, we extended the Manticore runtime system with the ability to restore determinism at runtime when a speculative thread gets cancelled.  Additionally, we also formally proved that our proposed solution correctly preserves a deterministic semantics.  We did this by providing an operational semantics using the Coq theorem prover, and proved that this language was observably equivalent to a language without speculative paralleism.  By combining these features, programmers can now write richer and more expressive programs and still enjoy the benefits of working in a deterministic parallel language.  This work was recently presented at the International Symposium on the Implementation and Application of Functional Languages (IFL'14) and is being prepared for post-symposium peer review.  

Over the summer, I was fortunate enough to attend the Oregon Programming Languages Summer School, which is a two week lecture series on various topics in programming languages research.  This allowed me to meet and network with many other PhD students in my field as well as some of the most highly respected and accomplished researchers who were giving lectures.  Being able to learn about other students' research experiences gave me perspective and a better understanding of how to approach research problems and develop a thesis topic.  During the lectures, I learned about many other ongoing research efforts in the field, and have been applying concepts regarding formal verification to my current research. In addition to the summer school, I also received a travel grant to attend the ACM SIGPLAN Symposium on Principles of Programming Languages (POPL'14) in San Diego, which also allowed me to meet and network with a number of other researchers and learn more about ongoing research efforts in the field.  

During my time as an undergrad I was fortunate to have the pleasure of working with two research groups at the University of Minnesota for which I was nominated for the ``Computing Research Association Outstanding Undergraduate Research'' award. My interest in research began when I took a software engineering class that was taught by Dr. Eric Van Wyk. Our class project was to implement a very basic compiler. I became fascinated by programming languages and was eventually offered an opportunity to work with his research group.

While working with Professor Van Wyk, we explored how to write compilers in such a way that new features can be added to the language in an extensible and composable way. Our approach used a form of rewriting, known as \emph{forwarding} which provides certain guarantees that a new language extension will compose with the other existing extensions. More specifically, we are able to guarantee that all semantic analyses are defined in the language extension, and that we will be able to parse a given program without any conflicts with the new extensions. I was able to publish my contributions to the project in a workshop on domain specific languages, and presented it last year in Denver.  

This work has had tremendous impacts on my life. Prior to this, I had almost no interest in pursuing graduate school; attaining a PhD and pursuing a career in research seemed like an unachievable goal. However, Dr. Van Wyk showed me that this is far from true. In my time working with Dr. Van Wyk I have learned that in order to be a successful researcher it takes passion for the material, creativity, and a good work ethic. Dr. Van Wyk has served as an invaluable resource to me and a mentor. I credit him with sparking my interest in academia and helping me to grow as an independent researcher. It is a goal of mine to eventually provide these same resources to others.

A second experience that had a substantial impact on my decision to pursue graduate school is the research that I did with Dr. Vipin Kumar and his PhD student James Faghmous in the field of data mining. In this project, I worked closely with James, a sixth year PhD student at the time, on implementing algorithms for detecting and tracking ocean eddies, which are spinning pools of water that are responsible for transporting heat, salt, energy, and nutrients throughout the ocean.

Ocean eddies have been found to have a number of impacts on marine ecosystems, so understanding their behavior is essential. One contribution that we made to this field was on the identification of ocean eddies. Our methods took an image processing approach where we analyze sea surface height data and try to identify eddy-like structures.  In this work I was responsible for implementing the proposed idea, as well as all of the code that was used to test and evaluate our methods. In addition to this, I was a part of the many discussions in which these ideas were developed and played a large role in designing the evaluation of our algorithm. This work was published in the IEEE International Conference on Data Mining (ICDM'13), in which I was able to also be involved in the writing of the paper.

A second project that I worked on with this group was on the tracking of ocean eddies.  Our approach applied a method known as multiple hypothesis tracking where we maintain a tree of possible paths an eddy could have taken, and after some number of time steps we make a decision as to which path is most likely to have been the real path of the eddy. This way, we can defer our decision of tracking until we have more information available. This approach proved to be very successful in correctly tracking ocean eddies and was published in the AAAI Conference on Artificial Intelligence (AAAI'13).

Although artificial intelligence and data mining are not my primary research interests, I have learned a tremendous amount about research in having worked with this group. First of all, I have experienced first hand what it takes to produce a top tier conference publication. Second, I have learned how to work in a group setting and collaborate with researchers from other universities. Professor Kumar's research group currently holds the NSF Expeditions in Computing grant along with five other universities, so communicating effectively with researchers from all over the country is something that I experienced from the beginning.
 
Throughout my time working with James, I learned a great deal about what it takes to become a successful researcher in general. James had taken me under his wing in my time working with him. He always encouraged me to go outside my comfort zone and included me in activities such as conference calls with other researchers and involved me in the writing process of our two publications. It was interesting to learn from and get the perspective of someone who was just about to finish the entire experience and obtain his PhD.

My experiences in research and academia have benefited many others from a diverse community as well. In my third year of undergrad I was required to do some volunteer work for a course that I was taking. I decided to serve as a tutor working with adults pursuing their GED and Adult Basic Education classes. The majority of our beneficiaries were newly arrived immigrants with limited English-speaking abilities. I was only required to stay for one semester, however, I enjoyed teaching so much that I ended up continuing to volunteer there until I graduated (an additional year and a half). This experience has stimulated my interest in pursuing teaching as a career. The feeling that one gets when someone they have been tutoring gets their first A on a test, or finally understands a concept they haven't been able to get on their own is like no other, and is something that I would like to pursue on a larger scale.  RIT has recently allowed graduate students to teach courses over the Winter term, and I hope to take advantage of this opportunity in the near future.  

In more of a research setting, I am currently involved in a number of master's students' projects who are working with my advisor on the MLton compiler.  I have been regularly attending their research meetings and encouraging them to pursue research as a future career.  For me, getting into research has been one of the best things to happen to me, and so I feel compelled to persuade others to do the same. In my own research projects, I am very interesting in finding undergrads to work with more closely in an effort to introduce them to something that I have come to know and love.

There are many things that have happened in my past that have helped to prepare me for a career in research. I feel confident that I have the abilities to make it as an independent researcher in the field of computer science, however, this NSF fellowship will give me the freedom to shape my own research agenda, something that I believe will provide a creative environment for me to produce transformative research.

\begingroup
\renewcommand{\section}[2]{}%
{\footnotesize
\begin{thebibliography}{}


\bibitem{icfp14} L. Bergstrom, M. Fluet, {\bf M. Le}, J. Reppy, and N. Sandler. Practical and Effective Higher-Order Optimizations. In ICFP'14

\bibitem{wolfhpc} {\bf M. Le} , K. Williams, and E. Van Wyk. A Composable Domain Specific Language Extension for Spatio-Temporal Data Mining. In WOLFHPC'13

\bibitem{AAAI13} J. H. Faghmous, M. Uluyol, L. Styles, {\bf M. Le}, V. Mithal, S. Boriah, and V. Kumar. Multiple Hypothesis Object Tracking for Unsupervised Self-Learning: An Ocean Eddy Tracking Application. In AAAI'13

\bibitem{ICDM14} J. H. Faghmous, {\bf M. Le}, M. Uluyol, S. Chaterjee, and V. Kumar. Parameter-Free Spatio-Temporal Pattern Mining to Catalogue Global Ocean Dynamics. In ICDM'13


\end{thebibliography}
}
\endgroup


\end{document}












