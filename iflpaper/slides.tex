\documentclass[mathserif]{beamer}

\mode<presentation> {

\usetheme{Madrid}

}
\usepackage{stmaryrd}
\usepackage{listings}
\usepackage{graphicx} % Allows including images
\usepackage{booktabs} % Allows the use of \toprule, \midrule and \bottomrule in tables

\newcommand{\bis}[1]{\begin{itemize}\setlength{\itemsep}{#1}}

\begin{document}

\title[RPA]{Combining Shared State with Speculative Parallelism in a Functional Language}

\author[Matt Le]{Matt Le \\ Advisor: Matthew Fluet}

\institute[RIT]{Rochester Institute of Technology}
\date{\today} % Date, can be changed to a custom date

%Title
\begin{frame}
\titlepage
\end{frame}


%Motivation
\begin{frame}{Motivation}
\bis{.5cm}
\item Functional languages do not allow mutable state
\item Lack of mutable state provides deterministic parallelism
\item Unfortunately, some applications can be more efficiently encoded with shared state
\end{itemize}
\end{frame}

%Ivars
\begin{frame}{IVars}
\bis{.5cm}
\item May only be written to once
\item Allows threads to share information
\item Found in parallel functional languages such as Id, CnC, and Parallel Haskell
\end{itemize}
\end{frame}

%Speculative Parallelism
\begin{frame}[fragile]{Speculative Parallelism}
\bis{.5cm}
\item Method for parallelizing applications with sequential dependencies
\item Guess the output of the dependent computation
\item Run the second computation in parallel.  If we guess wrong, cancel and restart the second thread
\item Graph Coloring:
\begin{itemize}
\item Guess whether or not the graph is colorable using an approximate polynomial time algorithm
\item Confirm the result using an exhaustive approach
\end{itemize}

\end{itemize}


\end{frame}

\begin{frame}{Speculation and Shared State}
\begin{itemize}
\item  Speculative parallelism and IVars have separately been added to functional languages
\item In this work we combine both features in Manticore
\end{itemize}
\end{frame}

%Nondeterminism
\begin{frame}[fragile]{Nondeterminism}

\begin{lstlisting}[language=ML, mathescape, keywordstyle=\color{blue}, morekeywords={pval, pcase},basicstyle=\footnotesize, xleftmargin=7.0ex]
let exception E
    val i = IVar.new()
    val _ = (|raise E, IVar.put i 10|) 
    	          handle E => ((), ())
in get i
end
\end{lstlisting}

\end{frame}

%Runtime Support
\begin{frame}{Runtime Support}
\bis{.5cm}
\item Have threads log their reads, writes, and threads that are forked
\item If a thread is canceled, ``undo'' its effects
\item Restart all threads who read rolled back values and their transitive closures
\end{itemize}
\end{frame}

%Formal Semantics
\begin{frame}{Formal Semantics}
\bis{.5cm}
\item Extend the Par Monad semantics with speculation

\item Prove $H; e \rightarrow^*_s H' e' \iff \mathcal{E}\llbracket H; e\rrbracket \rightarrow^*_p \mathcal{E}\llbracket H'; e' \rrbracket$

\end{itemize}
\end{frame}

%Coq Formalization
\begin{frame}{Coq Formalization}
\bis{.5cm}
\item Using the Coq theorem prover to formalize portions of this work

\item Independence Lemma: If $H; T_1\;|\;T_2 \rightarrow_s H'; T_1' \; | \; T_2 \xrightarrow{pure}_s H'; T_1'\;|\;T_2'$ then $H; T_1\;|\;T_2 \xrightarrow{pure}_s H; T_1 \; | \; T_2' \rightarrow_s H'; T_1'\;|\;T_2'$

\begin{itemize}
\item Requires consideration of 361 cases
\end{itemize}

\end{itemize}
\end{frame}

%Implementation
\begin{frame}{Implementation}
\bis{.5cm}
\item Added IVars to Manticore
\item Extended the runtime system with the proposed rollback mechanism
\end{itemize}
\end{frame}

%Evaluation
\begin{frame}{Preliminary Evaluation}
\begin{itemize}
\item 5\% overhead for non-speculative reads and writes
\item Minimal overhead on rollback benchmark (0.1833 vs. 0.1834 seconds)
\item Near linear speedup (3.385X) for register allocation benchmark
\end{itemize}

%\begin{figure}
%\centering
%\includegraphics[scale=.3]{SyntheticResults.pdf}
%\caption{Rollback Benchmark Results}
%\end{figure}
\end{frame}

%Future Work
\begin{frame}{Future Work}
\bis{.5cm}
\item Work on improving the efficiency of the runtime system
\item Perform a more in depth evaluation
\item Generalize methods to LVars
\end{itemize}
\end{frame}

%Questions?
\begin{frame}
\Huge{\centerline{Questions?}}
\end{frame}


\end{document} 