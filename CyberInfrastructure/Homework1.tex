\documentclass[11pt]{article}


\usepackage[margin=1in]{geometry}
\usepackage{listings}
\usepackage{color}
\usepackage{program}

\title{Cyberinfrastructure Foundations: Homework 3}
\author{Matthew Le}

\begin{document}
\maketitle


{\bf Question 1)} In the context of parallel computing, a shared memory architecture is one where a single CPU has multiple cores running in parallel that communicate with each other by reading and writing to main memory, which is shared by all the cores.  A cluster architecture, is one which is distributed over multiple machines, where parallel processes communicate by sending messages to one another.  Hybrid parallel computers are those that employ both distributed and shared memory architectures.  It is essentially a distributed architecture where each machine that is connected through the network has multiple shared memory multiprocessors. \\
{\bf Question 2)}
\begin{figure}[h]
\centering
\begin{tabular}{|c|c|c|c|c|}
\hline
N & 1 & 2 & 3 & 4 \\\hline
40 &  0.9639   & 1.5679  &  1.9564   & 2.2595 \\\hline
 160 &   0.9706   & 1.8295   & 2.5355   & 3.1589 \\\hline
360 &   0.9819    & 1.8705   & 2.6839   & 3.4426\\\hline
640 &    0.9692    & 1.9086    &2.7298    &3.5691\\\hline
 1000&   0.9760   &  1.9196    &2.7726    &3.6202\\\hline
\end{tabular}
\caption{Speedup (Relative to Sequential)}
\end{figure}


\begin{figure}[h]
\centering
\begin{tabular}{|c|c|c|c|c|}
\hline
N & 1 & 2 & 3 & 4 \\\hline
40 & 96.3909  & 78.3971 &  65.2126  & 56.4876\\\hline
160 &   97.0555&   91.4754&   84.5180&   78.9729\\\hline
360 &   98.1940   &93.5246  & 89.4639   &86.0652\\\hline
640 &   96.9186  & 95.4310  & 90.9942  & 89.2266\\\hline
1000 &   97.5952  & 95.9813  &  92.4204 &  90.5038\\\hline

\end{tabular}
\caption{Efficiency (Relative to Sequential)}
\end{figure}

\begin{figure}[h]
\centering
\begin{tabular}{|c|c|c|c|c|}
\hline
N & 1 & 2 & 3 & 4 \\\hline
40 & 1.0384    & 0.6285  &  0.4986    &0.4283\\\hline
 160 &   1.0305 &   0.5437&    0.3906  &  0.3123\\\hline
 360 &   1.0184   & 0.5333   & 0.3708  &  0.2885\\\hline
  640 &  1.0318   & 0.5232    &0.3653   & 0.2791\\\hline
   1000 & 1.0247   & 0.5205   & 0.3600  &  0.2755\\\hline
   
   \end{tabular}
   \caption{EDSF (Relative to Sequential)}
   \end{figure}

\begin{figure}[h]
\centering
\begin{tabular}{|c|c|c|c|c|}
\hline
Cores & Sizeup (15,000) & Eff (15,000) & Sizeup (20,000) & Eff (20,000) \\\hline
1   &     0.975      &   0.975     &   0.975 & 0.975   \\\hline
 2    &     1.941      & 0.970       &  1.941  & 0.971\\\hline
  3   &    2.826   & 0.942       & 2.827   & 0.942\\\hline
   4    &   3.733     &    0.933    &     3.735  & 0.933\\\hline

\end{tabular}
\caption{Sizeup Information (Relative to Sequential)}
\end{figure}

\begin{figure}[h]
\begin{program}
\PROC |cos|(x) \BODY
	x = x \; \%\; (2*3.14159);
	sign = true;
	i = 2;
	sum = 1;
	denom = 1;
	|repeat| 
		denom = denom * (i-1) * i;
		term = exp(x, 2 * i) / denom;
		\IF (sign) 
		\THEN sum = sum + term;
		\ELSE sum = sum - term;
		\FI
	i = i+2;
	sign = !sign;
	|until|(abs(term / sum) < 0.001)

\END

\PROC |seqCos|() \BODY
	results = new \; array;
	\FOR i = 0.0 \; |to| \; 15000 \; |by| \; 0.1\DO
		results[(int) i * 10] = cos(i);
	\END
	|return| \; results;
	\END

\PROC |parCos|() \BODY
	results = new \; array;
	|par| \FOR i = 0.0 \; |to|\; 15000 \; |by| \; 0.1 \; \DO
		results[(int) i * 10] = cos(i);
	\END
	\END
\end{program}
\end{figure}

\begin{figure}[h]
\begin{program}
\PROC |mailman|(world) \BODY
	start = world.start();
	end = world.end();
	probs = [1,1,1,1,2,2,3,4];
	x = 0; y = 0;
	\FOR i = start \; |to| \; end \DO
		|switch|(probs[randInt(1,4)])\{
		|case|  \;0 : x = x + 1; break;
		|case| \;1 : x = x - 1; break;
		|case| \; 2 : y = y-1; break;
		|case| \; 3 : y = y+1; break;
		\}
	\END
	resX = 0; resY = 0;
	xs = world.gatherXResult();
	ys = world.gatherYResult();
	\FOR i = 1 \; |to| \; length(xs) \DO
		resX = resX + xs[i];
		resY = resY + ys[i];
	\END
	return(resX, resY);
\end{program}
\end{figure}

\end{document}






